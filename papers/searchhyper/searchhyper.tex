\documentclass{../acm_proc_article-me11_tweaked}

\begin{document}

\conferenceinfo{\textit{MediaEval 2013 Workshop,}}{October 18-19, 2013, Barcelona, Spain}

\title{}

%
\def\sharedaffiliation{%
\end{tabular}
\begin{tabular}{c}}
%

\numberofauthors{4}
\author{
% 1st. author
\alignauthor
Jonathon S. Hare\\
       \email{jsh2@ecs.soton.ac.uk}
% 2nd. author
\alignauthor
Sina Samangooei\\
       \email{ss@ecs.soton.ac.uk}
% 3rd. author
\alignauthor
David P. Dupplaw\\
			\email{dpd@ecs.soton.ac.uk}
% 4th. author
\and
\alignauthor
Paul H. Lewis\\
       \email{phl@ecs.soton.ac.uk}
\sharedaffiliation
       \affaddr{Electronics and Computer Science, University of Southampton, United Kingdom}
}

\maketitle
\begin{abstract}
\end{abstract}

% A category with the (minimum) three required fields
\category{H.4}{Information Systems Applications}{Miscellaneous}
%A category including the fourth, optional field follows...
\category{D.2.8}{Software Engineering}{Metrics}[complexity measures, performance measures]

\terms{}

\keywords{}

\section{Introduction}

\section{Architecture}
For the search sub-task, the system was to take a query and return a series of 
clips as results. This necessitated extracting relevant or otherwise 
appropriate sections from programmes. To facilitate this, individual 
programmes were generalised to functions of interest over time, where the real 
value at a given time indicated the instantaneous relevance of that point 
within the context of the current query. This paradigm permitted a modular 
approach to the system architecture, where individual modules can be applied 
to add, remove, or modify a set of timelines.

The modules developed operated by placing Gaussians on a timeline or by 
adjusting the factor by which a timeline was scaled (so indicating an overall 
increase in the relevance of one timeline compared to another). Where a 
Gaussian was to be defined or a scale factor increased, the value was 
calculated as 
\[a = S_m \alpha^{P_m}\]
where \(S_m\) was the scale factor for a module \(m\), \(\alpha\) was the 
module's base value for the operation, and \(P_m\) was the power factor for 
said module. This was done to provide the opportunity to control how much one 
module contributed to determining some section of a timeline as interesting, 
as well as the distribution of amplitudes of the Gaussians created by said 
module. See Table \ref{tbl:constants} for the values used for these variables 
when the runs were produced.

\subsection{Transcript module}
The transcript module was the most important component of the system, doing 
most of the heavy lifting when it came to determining the relevant sections 
of a programme. The module searched for keywords (taken from the query string) 
across all transcripts of a certain kind (LIMSI, LIUM, or subtitles). Matches 
were extracted from each transcript in turn, and a binary tree of these `hits' 
was then built using agglomerative clustering. The tree was walked, and when 
a cluster's separation (calculated as the distance between the average values 
of the cluster's left and right children) fell below a specified threshold, 
the cluster was used to build a Gaussian whose amplitude was calculated from 
\[\alpha = \frac{| W |}{| Q |} \sum_{w \in W} \operatorname{boost}(w) \operatorname{idf}(w)\]
where \(W\) was the set of keywords in the transcript, \(Q\) was the set of 
all possible keywords from the query, \(idf : W \to \mathbb{R}\) was a 
function mapping each keyword on to its inverse document frequency, and 
\(boost : W \to \mathbb{R}\) was a function mapping each keyword on to its 
boost in the query, if any. Additionally, the true amplitude was scaled by the 
normalised score returned by the search engine when searching for transcript 
documents matching the query. Thus the amplitude of the Gaussian captures the 
relevancy of all keywords in the cluster with respect to the document, as well 
as how completely the cluster covers the set of all possible query terms. The 
Gaussians were centered on the midpoint of the range covered by the cluster, 
and the parameter \(c\) of the Gaussian was chosen as one third of the 
temporal size of the cluster plus 60 seconds.

Originally a separate Gaussian was created for each keyword that matched in 
the transcript, but it was found that by examining clusters of keywords and 
taking into account how well each cluster covered the query, the quality of 
the results was improved.

\subsection{Other modules}
The synopsis and title modules increased the scale factor of any timelines 
whose synopses or titles matched the query by an amount derived from the 
search engine's score for the query in those fields and for the programmes 
corresponding to those timelines.

A channel filter module was implemented which performed some naïve NLP on the 
query: if it was found that a channel was mentioned in the query then any 
timelines corresponding to programmes on other channels were removed from the 
timeline set.

A concepts module looked in the query text and visual cues for known concept 
detections that could be added to timelines. The amplitude for the concept 
module's Gaussians was determined from the normalised confidence for each 
concept detection, and the width was a constant 5 seconds.

Additionally, another module worked purely visually, finding shots that were 
visually similar to existing shots with high confidence. For each programme, 
the most stable keyframe of each shot was extracted and SIFT features were 
calculated. These features were clustered using locality-sensitive hashing 
(henceforth LSH) and a graph was built whose vertices were keyframes such that 
any two keyframes were connected if the number of functions they collide under 
exceeded a given threshold. The module operated by finding sections of 
timelines corresponding to shots whose integrals exceeded a threshold (i.e. 
shots already deemed relevant by the preceeding modules), and added Gaussians 
centred on the shots whose keyframes were directly connected to this keyframe 
on the LSH graph. The base amplitude of the Gaussians was determined as the 
fraction of functions under which the two keyframes collided to the largest 
number of collisions, and the true amplitudes were additionally scaled by the 
integral of the shot from which the graph traversal originated. A constant 
width of 60 seconds was used.

\subsection{Tools and techniques}

\section{Results}

\section{Problems encountered}

\section{Further work}

\section{Conclusion}

\bibliographystyle{abbrv}
\bibliography{../bibliography}
\end{document}
