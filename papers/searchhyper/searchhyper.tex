\documentclass{../acm_proc_article-me11_tweaked}

\usepackage{tabu}
\usepackage{hyperref}
\usepackage[utf8]{inputenc}

\begin{document}

\conferenceinfo{\textit{MediaEval 2013 Workshop,}}{October 18-19, 2013, Barcelona, Spain}

\title{Southampton WAIS and the MediaEval 2013 Search and Hyperlinking Task}

%
\def\sharedaffiliation{%
\end{tabular}
\begin{tabular}{c}}
%

\numberofauthors{4}
\author{
% 1st. author
\alignauthor
Jonathon S. Hare\\
       \email{jsh2@ecs.soton.ac.uk}
% 2nd. author
\alignauthor
Sina Samangooei\\
       \email{ss@ecs.soton.ac.uk}
% 3rd. author
\alignauthor
David P. Dupplaw\\
			\email{dpd@ecs.soton.ac.uk}
% 4th. author
\and
\alignauthor
Paul H. Lewis\\
       \email{phl@ecs.soton.ac.uk}
\sharedaffiliation
       \affaddr{Electronics and Computer Science, University of Southampton, United Kingdom}
}

\maketitle
\begin{abstract}
\end{abstract}

% A category with the (minimum) three required fields
\category{H.4}{Information Systems Applications}{Miscellaneous}
%A category including the fourth, optional field follows...
\category{D.2.8}{Software Engineering}{Metrics}[complexity measures, performance measures]

\terms{}

\keywords{search, video, transcripts, agglomerative clustering, SIFT, 
          locality-sensitive hashing}

\section{Introduction}
A team from the University of Southampton's Web and Internet Science research 
group participated in the 2013 MediaEval Search and Hyperlinking challenge. 
A system was built to address the two tasks of retrieving video segments from 
an archive of programmes, given a textual query or an anchor segment from 
which to originate.\cite{mediaeval2013:searchhyper}

\section{Architecture}
For the search sub-task, the system was to take a query and return a series of 
clips as results. This necessitated extracting relevant or otherwise 
appropriate sections from programmes. To facilitate this, individual 
programmes were generalised to functions of interest over time, where the real 
value at a given time indicated the instantaneous relevance of that point 
within the context of the current query. This paradigm permitted a modular 
approach to the system architecture, where individual modules can be applied 
to add, remove, or modify a set of timelines.

The modules developed operated by placing Gaussians on a timeline or by 
adjusting the factor by which a timeline was scaled (so indicating an overall 
increase in the relevance of one timeline compared to another). Where a 
Gaussian was to be defined or a scale factor increased, the value was 
calculated as 
\[a = S_m \alpha^{P_m}\]
where \(S_m\) was the scale factor for a module \(m\), \(\alpha\) was the 
module's base value for the operation, and \(P_m\) was the power factor for 
said module. This was done to provide the opportunity to control how much one 
module contributed to describing (some section of) a timeline as interesting, 
as well as the distribution of values created by said module.

\subsection{Transcript module}
The transcript module was the most important component of the system, doing 
most of the heavy lifting when it came to determining the relevant sections 
of a programme. The module searched for keywords (taken from the query string) 
across all transcripts of a certain kind (LIMSI, LIUM, or subtitles). Matches 
were extracted from each transcript in turn, and a binary tree of these hits 
was then built using agglomerative clustering. The tree was walked, and when 
a cluster's separation (calculated as the distance between the average values 
of the cluster's left and right children) fell below a specified threshold, 
the cluster was used to build a Gaussian whose amplitude was calculated from 
\[\alpha = \frac{| W |}{| Q |} \sum_{w \in W} \operatorname{boost}(w) \operatorname{idf}(w)\]
where \(W\) was the set of keywords in the transcript, \(Q\) was the set of 
all possible keywords from the query, \(idf : W \to \mathbb{R}\) was a 
function mapping each keyword on to its inverse document frequency, and 
\(boost : W \to \mathbb{R}\) was a function mapping each keyword on to its 
boost in the query, if any. Additionally, the true amplitude was scaled by the 
normalised score returned by the search engine when searching for transcript 
documents matching the query. Thus the amplitude of the Gaussian captures the 
relevancy of all keywords in the cluster with respect to the document, as well 
as how completely the cluster covers the set of all possible query terms. The 
Gaussians were centered on the midpoint of the range covered by the cluster, 
and the parameter \(c\) of the Gaussian was chosen as one third of the 
temporal size of the cluster plus 60 seconds.

Originally a separate Gaussian was created for each keyword that matched in 
the transcript, but it was found that by examining clusters of keywords and 
taking into account how well each cluster covered the query, the quality of 
the results was improved.

\subsection{Other modules}
The synopsis and title modules increased the scale factor of any timelines 
whose synopses or titles matched the query by an amount derived from the 
search engine's score for the query in those fields and for the programmes 
corresponding to those timelines.

A channel filter module was implemented which performed some naïve NLP on the 
query: if it was found that a channel was mentioned in the query then any 
timelines corresponding to programmes on other channels were removed from the 
timeline set.

A concepts module looked in the query text and visual cues for known concept 
detections that could be added to timelines. The amplitude for the concept 
module's Gaussians was determined from the normalised confidence for each 
concept detection, and the width was a constant 5 seconds.

Additionally, another module worked purely visually, finding shots that were 
visually similar to existing shots with high confidence. For each programme, 
the most stable keyframe of each shot was extracted and SIFT features were 
calculated. These features were clustered using locality-sensitive hashing 
(henceforth LSH) and a graph was built whose vertices were keyframes such that 
any two keyframes were connected if the number of functions they collide under 
exceeded a given threshold. The module operated by finding sections of 
timelines corresponding to shots whose integrals exceeded a threshold (i.e. 
shots already deemed relevant by the preceeding modules), and added Gaussians 
centred on the shots whose keyframes were directly connected to this keyframe 
on the LSH graph. The base amplitude of the Gaussians was determined as the 
fraction of functions under which the two keyframes collided to the largest 
number of collisions, and the true amplitudes were additionally scaled by the 
integral of the shot from which the graph traversal originated. A constant 
width of 60 seconds was used.

\subsection{Hyperlinking}
The hyperlinking sub-task was viewed as an extension of the search sub-task, 
where the search query was an anchor as opposed to a string. Thus hyperlinking 
was addressed by slightly modifying the search architecture. 

A module was written to construct textual queries from anchors by extracting 
the portion of a transcript throughout the anchor segment. This allowed the 
use of the transcript, synopsis, title, and channel filter modules from the 
searcher architecture without any modification.

The concept module was re-written to find any concept detections during the 
anchor segment and to bring in other sections of video matching these 
concepts, whereas in the searcher architecture concepts were inferred from the 
query text.

The LSH SIFT graph module was re-written to operate by searching for similar 
frames starting with the frames in the anchor segment, whereas before this 
module was purely expansionary, operating on timeline sections that were 
already of high confidence.

\subsection{Tools and techniques}
In order to facilitate searching of transcripts, synopses, and programme 
titles, these features were indexed using Apache Lucene.

The Apache Commons Math library provided implementations of Gaussian functions 
and integrators which were used for constructing timelines and assessing the 
confidence of various segments, respectively.

Thomas Jungblut's implementation of agglomerative clustering 
(\url{https://github.com/thomasjungblut/tjungblut-math}) was used to 
enable clustering of transcript search hits.

SIFT features were extracted from keyframes using the OpenIMAJ 
library\cite{Hare:2011:OIJ:2072298.2072421} and Apache Hadoop.

\section{Results}

\section{Problems encountered}
A number of issues were encountered while developing the system.

The ground truth provided was not to a standard acceptable to facilitate 
automation of certain aspects of training of the system. Sometimes, the 
expected result was less relevant than other results which may be returned by 
the system. This was compounded by the fact that some queries were rather 
vague, and a number of results would adequately answer such queries. These 
factors placed a ceiling on the performance of the system, at least insofar as 
measurable by a metric based on sub-optimal ground truths.

The variety of different types of programmes eliminated the possibility of a 
single good all-round configuration for the system. For example, in a news 
programme keywords may pop up at regular intervals when headlines are read 
out, despite there being only a single short segment of said programme that 
contains the content the user is looking for. In contrast, a documentary may 
have multiple longer segments that are relevant, again with a different 
distribution of keywords. Ideally there would be some way to detect what 
genre of programme the system is extracting sections from, so as to maximise 
the potential relevance of such sections in the context of the query.

Originally sections were extracted from transcripts using Lucene's 
highlighting feature, but this was found to produce sub-optimal results: the 
highlighter tended to skew results towards the start of a transcript, and the 
length of the fragments returned was unpredictable, often extending much after 
the last keyword seen in said fragment. A manual approach to fragmentation 
using agglomerative clustering was used instead.

Additionally, there were a few issues with the provided data: typos in 
queries, unparseable subtitles, and missing concept detections, however these 
did not pose a systematic challenge.

\section{Further work}
One feature that was not implemented was person detection: ideally the system 
would be able to extract names from queries or faces/names from anchor 
segments and match these via a pre-computed database mapping between face 
detections and names. A good approximation could have been built using the 
provided metadata on characters and actors in each programme, and a system was 
envisioned that would allow arbitrary names to be queried against an online 
image datasource such as an image search engine, from which a detector could 
be trained and used to process the dataset live.

As previously mentioned, the performance of the system may be improved by 
categorising programmes and developing profiles for modules based upon these 
classifications.

\section{Conclusion}

\bibliographystyle{abbrv}
\bibliography{../bibliography}
\end{document}
