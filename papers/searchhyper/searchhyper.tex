\documentclass{../acm_proc_article-me11_tweaked}

\begin{document}

\conferenceinfo{\textit{MediaEval 2013 Workshop,}}{October 18-19, 2013, Barcelona, Spain}

\title{}

%
\def\sharedaffiliation{%
\end{tabular}
\begin{tabular}{c}}
%

\numberofauthors{4}
\author{
% 1st. author
\alignauthor
Jonathon S. Hare\\
       \email{jsh2@ecs.soton.ac.uk}
% 2nd. author
\alignauthor
Sina Samangooei\\
       \email{ss@ecs.soton.ac.uk}
% 3rd. author
\alignauthor
David P. Dupplaw\\
			\email{dpd@ecs.soton.ac.uk}
% 4th. author
\and
\alignauthor
Paul H. Lewis\\
       \email{phl@ecs.soton.ac.uk}
\sharedaffiliation
       \affaddr{Electronics and Computer Science, University of Southampton, United Kingdom}
}

\maketitle
\begin{abstract}
\end{abstract}

% A category with the (minimum) three required fields
\category{H.4}{Information Systems Applications}{Miscellaneous}
%A category including the fourth, optional field follows...
\category{D.2.8}{Software Engineering}{Metrics}[complexity measures, performance measures]

\terms{}

\keywords{}

\section{Introduction}

\section{Architecture}
For the search sub-task, the system was to take a query and return a series of 
clips as results. This necessitated extracting relevant or otherwise 
appropriate sections from programmes. To facilitate this, individual 
programmes were generalised to functions of interest over time, where the real 
value at a given time indicated the instantaneous relevance of that point 
within the context of the current query. This paradigm permitted a modular 
approach to the system architecture, where individual modules can be applied 
to add, remove, or modify a set of timelines.

\subsection{Transcript module}
The transcript module searched for keywords (taken from the query string) 
across all transcripts of a certain kind (LIMSI, LIUM, or subtitles). Matches 
were extracted from each transcript in turn, and a binary tree of these `hits' 
was then built using agglomerative clustering. The tree was walked, and when 
a cluster's separation (calculated as the distance between the average values 
of the cluster's left and right children) fell below a specified threshold, 
the cluster was used to build a Gaussian whose amplitude was
\[a = \frac{| W |}{| Q |} \sum_{w \in W} \operatorname{boost}(w) \operatorname{idf}(w)\]
where \(W\) was the set of keywords in the transcript, \(Q\) was the set of 
all possible keywords from the query, \(idf : W \to \mathbb{R}\) was a 
function mapping each keyword on to its inverse document frequency, and 
\(boost : W \to \mathbb{R}\) was a function mapping each keyword on to its 
boost in the query, if any. Thus the amplitude of the Gaussian captures the 
relevancy of all keywords in the cluster with respect to the document, as well 
as how completely the cluster covers the set of all possible query terms.

\subsection{Tools and techniques}

\section{Results}

\section{Problems encountered}

\section{Further work}

\section{Conclusion}

\bibliographystyle{abbrv}
\bibliography{../bibliography}
\end{document}
