\documentclass{../acm_proc_article-me11_tweaked}

\begin{document}

\conferenceinfo{\textit{MediaEval 2013 Workshop,}}{October 18-19, 2013, Barcelona, Spain}

\title{Identifying the Geographic Location of an Image with a Multimodal Probability Density Function}

%
\def\sharedaffiliation{%
\end{tabular}
\begin{tabular}{c}}
%

\numberofauthors{8}
\author{
\alignauthor
Jamie Davies\\
       \email{jagd1g11@ecs.soton.ac.uk}
\alignauthor
Jonathon S. Hare\\
       \email{jsh2@ecs.soton.ac.uk}
\alignauthor
Sina Samangooei\\
       \email{ss@ecs.soton.ac.uk}
\and
\alignauthor
John Preston\\
	\email{jlp1g11@ecs.soton.ac.uk}
\alignauthor
Neha Jain\\
	\email{nj1g12@ecs.soton.ac.uk}
\alignauthor
David P. Dupplaw
	\email{dpd@ecs.soton.ac.uk}
\sharedaffiliation
       \affaddr{Electronics and Computer Science, University of Southampton, United Kingdom}
}

\additionalauthors{Additional author: Paul H. Lewis (
email: {\texttt{phl@ecs.soton.ac.uk}})}

\maketitle
\begin{abstract}
    The geotagging of a photos location provides data that could be useful in a wide spectrum of applications. With the advance of digital cameras, and with many users exchanging their digital cameras for their GPS-enabled mobile phones, photographs annotated with geographical locations are becoming ever more present on photo-sharing websites such as Flickr. However there is still a wide majority of online content that is not geotagged, meaning that algorithms for efficient and accurate geographical estimation of an image are needed. We present a general model for using both textual metadata and visual features of photos to automatically place them on a world map. This forms the University of Southampton's entry for the MediaEval 2013 Placing task.
\end{abstract}

\section{Introduction and Motivation}

\section{Overall Methodology}

\section{Experiments}
\subsection{Run 1: Text+Visual, provided data}
\subsection{Run 2: Visual only, provided data}
\subsection{Run 3: Text only, provided data}
\subsection{Run 4: Text+Visual, bigger dataset}
\subsection{Run 5: Text+Visual, provided data with tag boosting}
\subsection{Results and Discussion}

\section{Conclusions and Future Work}

\section{Acknowledgments}
The described work was funded by the European Union Seventh Framework Programme (FP7/2007-2013) under grant agreements 270239 (ARCOMEM), and 287863 (TrendMiner).


\bibliographystyle{abbrv}
\bibliography{../bibliography}
\end{document}
