\documentclass{../acm_proc_article-me11_tweaked}

\begin{document}

\conferenceinfo{\textit{MediaEval 2013 Workshop,}}{October 18-19, 2013, Barcelona, Spain}

\title{Placing Task for MediaEval 2013}

%
\def\sharedaffiliation{%
\end{tabular}
\begin{tabular}{c}}
%

\numberofauthors{5}
\author{
% 1st. author
\alignauthor
Jonathon S. Hare\\
       \email{jsh2@ecs.soton.ac.uk}
% 2nd. author
\alignauthor
Sina Samangooei\\
       \email{ss@ecs.soton.ac.uk}
% 3rd. author
\alignauthor
David P. Dupplaw\\
			\email{dpd@ecs.soton.ac.uk}
% 4th. author
\and
\alignauthor
Paul H. Lewis\\
       \email{phl@ecs.soton.ac.uk}
% 5th. author
\and
\alignauthor
Jamie Davies\\
        \email{jagd1g11@ecs.soton.ac.uk}
\sharedaffiliation
       \affaddr{Electronics and Computer Science, University of Southampton, United Kingdom}
}

\maketitle
\begin{abstract}
    The geotagging of a photos location provides data that could be useful in a wide spectrum of applications. With the advance of digital cameras, and with many users exchanging their digital cameras for their GPS-enabled mobile phones, photographs annotated with geographical locations are becoming ever more present on photo-sharing websites such as Flickr. However there is still a wide majority of online content that is not geotagged, meaning that algorithms for efficient and accurate geographical estimation of an image are needed. We present a general model for using both textual metadata and visual features of photos to automatically place them on a world map. This forms the University of Southampton's entry for the MediaEval 2013 Placing task.
\end{abstract}

% A category with the (minimum) three required fields
\category{H.4}{Information Systems Applications}{Miscellaneous}
%A category including the fourth, optional field follows...
\category{D.2.8}{Software Engineering}{Metrics}[complexity measures, performance measures]

\terms{}

\keywords{}

\section{Introduction}


\bibliographystyle{abbrv}
\bibliography{../bibliography}
\end{document}
